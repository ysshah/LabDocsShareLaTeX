\documentclass{../signatures}

\begin{document}

\noindent{\LARGE OPT Pre-Lab, Mid-Lab, Checkpoints}
\hfill
Student's Name \underline{\hspace{3cm}}
\begin{flushright}
Partner's Name \underline{\hspace{3cm}}
\end{flushright}

\noindent{\Large Pre-Lab Questions}

It is your responsibility to discuss this lab with an instructor on the first day of your scheduled laboratory period. This signed sheet must be included as the first page of your report. Without it you will lose 1/3 of a letter grade. You should think about and be prepared to discuss at least the following questions before you come to lab:
\begin{enumerate}
    \item What is the general principle of optical pumping? Go over your derivation of the Breit-Rabi formula and the values of the Lande g-factors of the hyperfine energy levels of $^{85}$Rb and $^{87}$Rb. Draw qualitative energy-level diagrams for $^{85}$Rb and $^{87}$Rb showing the fine, hyperfine, and Zeeman splittings. How do the Lande g-factors affect the ordering of the Zeeman levels? Show the transitions between these levels that are important to this experiment. Include these drawings in your write-up. For our rubidium system, what is the pumping process? Where is the pumped level? Where is the RF transition?
    \item Why do we modulate (vary sinusoidally) the external magnetic field? How would we take data if the magnetic field were not modulated?
    \item In this experiment, how will you determine the resonance frequency? How can you best estimate the error? Will the modulation amplitude affect your result? What data will you take, and what plots will you make?
\end{enumerate}

\prelabsignature

\vspace{1cm}

\noindent{\Large Mid-Lab Questions}

On day 2 of this lab, you should have successfully produced a plot of frequency versus current for at least one rubidium isotope, and have made an estimate of the earth’s magnetic field. Show them to an instructor and ask for a signature.\\

\midlabsignature{second}

\vspace{1cm}

\noindent{\Large Checkpoints}
\begin{enumerate}
    \item Displaying your knowledge of the Function Generator.

    \signature

    \item Finding resonance conditions and the symmetries associated with the signal and reasoning behind viewing modes.

    \signature

    \item Determining the statistical error in your measurement technique.

    \signature
\end{enumerate}

\end{document}
